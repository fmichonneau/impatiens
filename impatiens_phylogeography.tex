
\documentclass{article}
\usepackage{authblk}

\title{Cryptic and not-so-cryptic species in the complex ``\textit{Holothuria
    (Thymiosycia) imaptiens}'' (Forrsk\r{a}ll, 1775) (Echinodermata:
  Holothuroidea: Holothuriidae)} \author[1]{Fran\c{c}ois
  Michonneau\thanks{Corresponding author: francois.michonneau@gmail.com}}
\author[1]{Gustav Paulay} \affil[1]{Florida Museum of Natural History,
  University of Florida, Gainesville, FL 32611-7800, USA}

\renewcommand\Authands{ and }


\begin{document}

\maketitle

\begin{abstract}
Blabla
\end{abstract}

\section{Introduction}

The failure to distinguish reproductively isolated species can stem from either
the choice of an inappropriate set of characters to distinguish species
otherwise morphologically variable, or from the absence of variation in
morphology but differentiation in other traits (e.g., ecology, chemical cues)
allows them to remain reproductively isolated \cite{Knowlton1993}.

For morphological characters to be appropriate to distinguish between
populations of reproductively isolated individuals, they require to evolve at a
rate of the same par as populations become species. They also must exhibit less
intraspecific than interspecific variation. Traits that are under sexual
selection usually meet these criteria. When species use visual clues to identify
their mates, taxonomists can use the same characters to discrimminate between
these populations of reproductively isolated individuals. However, in situations
where mate recognition does not involve visual clues (e.g., chemical
communication, sperm/egg recognition proteins), whether a particular
morphological trait evolves at the appropriate pace to distinguish between
species is governed by other evolutionary forces. These traits are, at best, a
by-product of reproductive isolation that do not guarantee to be effective in
discrimminating species.

Most marine invertebrate taxonomy results from the study of dry or preserved
specimens. With these methods, many delicate morphological characters and very
often color patterns are lost. Therefore, in groups characterized by hard parts,
taxonomists have relied on these skeletal elements to find characters that allow
them to differentiate species. For instance, in sea cucumbers, in particular in
the Holothuriidae, species-level taxonomy has entirely been focused on the shape
of ossicles, microscopic calcareous secretions found in many tissues. At least
until the late 1980's most species descriptions would ignore intraindividual
variation by figuring only ossicles found in the body wall of the species, and
generally ignored any intraspecific variation by illustrating the ossicle
assemblage of a single individual.

% who were the first authors to illustrate ossicles?

Forrsk\r{a}ll described \textit{Holothuria impatiens} in 1775 from the
collections he was making in Suez, Egypt. The original description is vague, and
could fit many other holothuriid species of the area, he did however include a
drawing of the species. Since then, the range of \textit{H. impatiens} has been
extended, and the species is now thought to be occurring in the Red Sea, the
entire Indo-West Pacific region, the Caribbean and the Mediterrean seas.

\textit{Holothuria impatiens} is the type species of the subgenus
\textit{Thymiosycia} one of the 18 subgenera that are forming the genus
\textit{Holothuria}. The subgenus \textit{Thymiosycia} was first erected by
Pearson \cite{Pearson1915} to include all holothuriid being characterized by the
absence of ``true pedicels'' and anal teeth, and the presence of ``papillae
scattered all over the body, generally on eminences''; and designed
\textit{H. impatiens} as the type species for the subgenus. In the most recent
revision of the family, Rowe \cite{Rowe1969} redefined \textit{Thymiosycia} to
only include species which are characterized by having well developed, regular
and symmetrical tables and buttons. The ossicle assemblage of
\textit{H. impatiens} therefore exemplifies the holothuriid ossicles.

\textit{Holothuria impatiens} is a common, locally fairly abundant, species
which can be found under rocks in shallow waters (in particular in lagoons and
back-reef habitats) but has been recorded down to 158~m (\cite{Samyn2013}).
Because of its ubiquituous distribution, it is one of the more studied species
of holothuriid with studies on its reproduction \cite{Harriot1985}, its
Cuvierian tubules \cite{Flammang2002}, \cite{Becker+Flammang2010}; its toxicity
\cite{Bakus1974}; its feeding preferences \cite{Roberts1982}; the chemical
composition \cite{Hampton1958}, statistical analysis \cite{Hampton1959},
ontogenic changes in its ossicles \cite{Cutress1996}; and its parasites
\cite{Martens1994}.  It has also been included in molecular phylogenies (e.g.,
\cite{Lacey2005}, \cite{Honey-Escandon2012}). \textit{Holothuria impatiens} is a
low-value commercial species but is reported to be fished in the Eastern Pacific
\cite{Toral-Granda2008}, Madagascar \cite{Conand2007}, and Palau
\cite{Pakoa2009}.

Despite its relative biological and commercial importance, the variation
observed in their color patterns and reported by previous workers (e.g.,
\cite[p.178]{Clark1921}, \cite{Rowe+Richmond2004}) has yet to be
investigated. In the present study, using a combination of morphological,
genetic and ecological data, we show that \textit{Holothuria impatiens}
represents a complex of 12 species. If some of these species can be
distinguished relatively easily from their live appearance, others are harder to
tease apart. This, in combination with the fact that several of these species
co-occur, and sometimes even share the habitat, has contributed to the failure
to identify the species limits in this complex.

Understanding the dynamics of diversification in the sea is challenged by the
fact that most marine invertebrates have a highly dispersive larval stage and
that oceans in general and the Indo-Pacific in particular do not seem to have
clear geographical barriers to dispersal. Yet, while some species have broad
geographical ranges encompassing most of the IWP province, others are more
limited, and some closely related species have allopatric distributions. The
frequent overlap in species distribution across the IWP challenges our current
understanding of the role of geographical isolation in speciation. It is
generally accepted, at least on land, that allopatry is the most common mode of
speciation. However, the combination of broadly overlapping ranges and high
diversity leaves seemingly few opportunities for allopatry in the
sea. Furthermore, recent geological and molecular evidence suggests that most of
the modern reef fauna was shaped during the Miocene. To account for this
apparent paradox, several alternative models have been proposed. One category
downplays the importance of the absence of gene flow (allopatry) and emphasizes
the role of selection (along environemental gradients, sexual selection, host
specificity) as the main driver behind reproductive isolation (see
\cite{Bowen2013}). Another category of models proposes that speciation occurred
in allopatry, but since then, important changes in species ranges and
distributions have obscurred the barriers responsible for the initiation of the
reproductive isolation.

Inferring the mode of speciation from current distributions in the Indo-Pacific
is most likely misleading as modern ranges in the Pacific are necessarily
relatively recent as sea levels have fluctuated greatly (ca. 100-150~m) and
repeatedly (ca. 100~k.y. periodicity) during the late Tertiary and early
Quarternary. These fluctuations have led to important habitat modifications and
changes in oceanic current patterns with consequences in connectivity,
population sizes and species ranges. These changes have made species
distributions in the Pacific highly dynamic (\cite{Paulay1990}) and are, in most
cases, too recent compared to the time frame needed for speciation. 
% However, speciation can be very rapid, e.g., Asterinid from NSW Byrne2012
For instance, the latest minimum sea level (120 to 135~m below present) was
18,000 years ago. During this period, circulation between Pacific and Indian
Oceans was significantly restricted, the Red Sea was most likely disconnected
from the Indian Ocean, and the South China sea was land-locked
(\cite{Veron1992}). Therefore, current distribution patterns reflect only
partially the biogeographical context under which the diversification of modern
reef-associated fauna has occurred.

The consequences of these fluctuations in sea level on the populations might
however still be visible in the present genetic diversity of the modern
fauna. For instance, the low sea levels caused an important reduction of the
connectivity between the Indian and Pacific oceans providing repeated
opportunity for population divergence between the two ocean bassins. While some
species retained a genetic signature of this episode of low connectivity in
their mitochondrial DNA (e.g., xxxxxx), many other lineages (including the most
widespread form of \textit{Holothuria impatiens}) show little geographic genetic
structuring across their range. This latter pattern can be explained by (1)
maintained connectivity during these low sea stands, as both oceans were still
connected (\cite{Voris2000}); (2) one of the two lineages was lost because of
local extinction or selective sweep; (3) the species recently expanded their
ranges and colonized one of the bassin very recently.


% Allopatry without clear geographic barriers (Diadema setosum, Lessios 2001)

% how to include peripatric speciation?
% say something about how these 2 models are not mutually exclusive

\section{Methods}

\subsection{Sampling}

Specimens used for this study are housed in the collections of the Florida
Museum of Natural History at the University of Florida, Gainesville, FL, USA. A
few additional tissue samples with voucher hosted in other institutions or with
no voucher were obtained through collaborators (Table~\ref{tab:voucherInfo} for
details).

Specimens were collected at low tide, on snorkel, on SCUBA or by dredging. Most
specimens were photographed while alive \textit{in situ} or in the lab,
anesthetized in a 1:1 solution of sea water and 7.5\% solution of magnesium
chloride hexahydrate, before being fixed in 75\% ethanol. When possible
tentacles were clipped before fixation to be used in DNA extractions.

Two hundred and fifty three specimens were examined morphologically and 216 were
used for molecular analyses. These specimens were collected across the entire
known range of \textit{H. impatiens}: Mediterranean Sea, Carribean Sea, Red Sea,
Tropical Indian and Pacific Oceans.

% TODO: make specimen table

\subsection{DNA extraction and amplification}

DNA was extracted using either Invitrogen\texttrademark\ DNAZol\textregistered\
or Omega Bio-Tek\texttrademark\ E.Z.N.A\textregistered\ Mollusc DNA kit following
manufacturer recommendations. 

In this study we amplified the mitochondrial markers COI, 16S, ATP6 and the
nuclear markers histone 3 (H3a), 18S, ITS1-5.8S-ITS2, c0036 and c0775. These
last two markers have been developed from a 454 run on genomic DNA from
\textit{H. edulis}, and additional details will be provided in a future study
(\cite{MichonneauInPrep}). Primers, and PCR conditions used are provided in
Table~\ref{tab:pcrConditions}. Because of the length of ITS1-5.8S-ITS2, we used
the sequencing primer fm-5.8S-f. The primers fm-ITS-f is the reverse complement
sequence of the primer 18S-1708R (\ref{}) and fm-ITS-r is the reverse complement
of LSUFW1 (\ref{}).

We conducted PCR in 25µL reactions using either 15.4µL of water, 2.5µL of
Sigma-Alrich \textregistered 10X PCR buffer, 2.5µL of dNTP, 2µL of $MgCl_2$, 1µL
of the forward primer, 1µL of the reverse primer, and 0.1µL of
Sigma-Aldrich\textregistered Jumpstart\texttrademark Taq DNA polymerase or the
Promega GoTaq MasterMix following manufacturer recommendations.

Sequencing of PCR products was performed by the Interdisciplinary Center for
Biotechnology Research at the University of Florida.

\begin{table}
  \centering
  \begin{tabular}{ | l | p{5cm} | p{5cm} | p{1cm} | p{1cm} | p{1cm} |}
    Marker        & Primer Forward ($5'-3'$) & Primer Reverse ($5'-3'$)  & Annealing temperature &  Number of
    cycles & Reference \\
    COI           & COIceF ACT GCC CAC GCC CTA GTA ATG ATA TTT TTT ATG GTN ATG CC & COIceR
    TCG TGT GTC TAC GTC CAT TCC TAC TGT RAA CAT RTG & 42 & 40 & \cite{Hoareau2010} \\
    16S & 16SAR CGC CTG TTT ATC AAA AAC AT & 16SBR GCC GGT CTG AAC TCA GAT CAC
    GT & 52 & 35 & \cite{Arndt1996} \\
    ATP6 & ATP6f GGA CAA TTT TCC CCA GAC CT & ATP6r GGT GAA GAG GGT GTT GAT GG &
    42 & 40 & this study \\
    28S & LSUFW1 AGC GGA GGA AAA GAA ACT A & LSUREV2 ACG ATC GAT TTG CAC GTC AG
    & 42 & 40 & \cite{} \\
    ITS1-5.8S-ITS2 & fm-ITS-f AGG TGA ACC TGC AGA TGG ATC A, fm-5.8S-f CGT CGA TGA AGA ACG
    CAG YW & fm-ITS-r TAG TTT CTT TTC CTC CGC T & 45 & 40 & This study, \cite{}, \cite{} \\
    c0036 & c0036f TAA CGA CGG ATC TCA CGG AG & c0036r AAT AAT GCT GGC GTG ACG
    TC & 42 & 45 & This study \\
    c0775 & c0775f GCT CTT CGT TCA ATT TAT CTC GC & c0775r GGG ATG CAG TTT GTC
    GAG TG & 42 & 40 &This study \\
  \end{tabular}
  \label{tab:pcrConditions}
\end{table}

\subsection{Genetic diversity}
% look for locality with highest number of unique haplotype 

\subsection{Morphological analysis}

To compare intra- and inter-specific variation in ossicle size in the dorsal
body wall, we measured various characteristics of buttons and tables composing
the ossicle assemblage of \textit{Holothuria impatiens}
(Fig.\ref{fig:ossicleMeasurements}).

For 3-5 individuals of each ESU, we measured characteristics of at least 10
buttons and 10 tables. Measurements were made from  photographs of the
ossicles at 400x magnification using ImageJ (\ref{ImageJ}).

Measurements were analyzed in a principal component analysis in R.

\subsection{Phylogeny and divergence time estimation}

We estimated the phylogenetic relationships and the timing of the divergence
among the different lineages of \textit{Holothuria impatiens} using BEAST v1.7.5
\cite{}. We used the closure of the isthmus of Panama to calibrate our
phylogeny. However, given that the sister ESU to the West Atlantic ESU is found
in the Galapagos and not in the Eastern Pacific, we used a log-normal prior
distribution on the stem node of these ESUs with a mean of 1.5 and a standard
deviation of 1 which corresponds to a mean of 4.482 My (0.865 -- 23.22 My 95\%
confidence interval) in real space.

We used PartitionFinder \cite{} to determine the best-fit partition scheme and
model of molecular of evolution for our dataset. We defined the data blocks such
that the three codon position for each protein coding locus (ATP6, COI, H3a,
c0036 and c0775) and each of 16S, ITS and LSU were individual partitions. We
selected the best-fit model based on the Bayesian Information Criterion. The
best fit model had 9 partitions. After some initial test runs in BEAST, we
modified slightly the model of molecular evolution to improve mixing of the MCMC
see Table \ref{tabPartitions}.

The analysis was run using independent strict molecular clock for each
locus. For COI we calibrated the clock to 0.037 substitutions per site per
million years following the review by Lessios \cite{Lessios2008}. All other
clocks were estimated with a log-normal distribution prior (mean -1, standard
deviation of 1). The tree prior was set to a Yule process and a random starting
tree. The Markov Chains were run for $50 10^6$ generations. The analysis was
repeated twice.

We used Tracer \cite{} to check that the MCMC chains had reach stationarity and
that the two independent runs were consistent. We ensured that enough of the posterior
distributions for each parameter was sampled by checking that the ESS values for
all parameters were above 200. After removing a 10\% burnin for each independent
run, the trees sampled were combined.


\subsection{GMYC}
To delineate putative species based on molecular data, we fit generalized mixed
Yule coalescent (GMYC) model (\cite{Pons2006}, \cite{Monaghan2009a}) to the COI
sequence data. This approach attempts to detect transitions in the branching
rate of an ultrametric tree corresponding to the expected increase in lineage
accumulation resulting from the shift to speciation to intraspecific coalescent
events. We fit the single (\cite{Pons2006}) and the multiple
(\cite{Monaghan2009a}) GYMC models, which allow either a single occurrence of
transition between inter- and intra-specific branching rate, or multiple
occurrences respectively. To determine the location of the occurrence, the GMYC
model is fit at different time points along the tree, and the time that provides
the best likelihood is selected. To assess whether including a transition in the
branching rate is statistically significant, a likelihood ratio test between the
model with no transition and the model with the transition is performed. The
same approach is used to determine if additional transitions improve the fit of
the model in the multiple GMYC approach.

The GMYC models were fit onto a the COI lineage genealogy estimated from
BEAST. We enforced a strict molecular clock, specified a Yule diversification
process, a GTR+I+G model of molecular evolution, and ran the chains for 10e6
generations. 


% Method in Pons+2006 and 
% Examples: Volger+2008 (COTS paper), Marshall+2011 (cicadas in New Zealand);
% Henrich+2010 (water beetles in Australia)
% See: Esselstyn+2012 as it includes simulation study of statistical power of
% this approach; Lohse2009 for a critic

\subsection{Skyline plots}



\section{Results}

\subsection{Diversity}
% how many clades
% how many specimens per clades


% cryptic species because indistinguishable, indistinguishable because similar
% looking or too much variation to make sense of it

% information about their biology gathered on the supposition it's a single
% species throughout its range

% from Volger+2012, most accurate COI divergence rate for echinoderms,
% 3.7(+/-)0.8%/Myr, from Lessios paper about Panamanian isthmus.

% look into splitstree (already installed into ~/Software folder), this allows
% to test for the robustness of signal in minimum spanning tree (see Volger+2012
% for details)

% look into \phi_{ST}, Fu's F, Tajima's D (reading the Arlequin manual)

% one of the reason, H impatiens has not been recognized as a species complex
% earlier, is due to the fact that some ESUs are widespread and others are
% overlapping. One of the reasons, these ESUs can overlap is that there was an
% ancestral widespread form which gave rise to the other species by parapatric
% speciation (e.g., as in Machel's hermit crabs), the alternative view is that
% each species recently extended their ranges. Should I try to use Lagrange? or
% Beast to reconstruct the phylogeographic history of the species to tease apart
% between these 2 hypotheses?
% to identify biogeographic regions, see if there are consistent phylogeographic
% patterns among ESUs

\section{Discussion}

% According to \cite{Renema2008}, the modern IWP fauna has its origin in the
% Miocene. Generation of hotspots associated with tectonic events, so importance
% of abiotic factors.

% Look into \cite{Lessios1999} to see how the timing in Eucidaris compares to
% the one H. impatiens

% Therefore, as it appears that the majority of speciation events are
% allopatric (\cite{Coyne2004}), recent changes in distribution and population
% sizes must be prevalent in the \textit{H. impatiens} complex to explain current
% distribution patterns. These changes have implications regarding the rapid
% evolution of reprodutive isolation barriers and ecological adaptations.

% Add something about how differences in colors among ESUs might be a result of
% the same type of proteins than GRP.

\bibliographystyle{plos2009}
\bibliography{impatiens_phylogeography}

\begin{table}
  \centering
  \begin{tabular} { | l | l | l |}
    1      & $TrN+I$      & $16S, ATP6_2$ \\
    2      & $GTR+I+G$    & $16Sc$        \\
    3      & $HKY+I+G$    & $ATP6_1$\\ 
    4      & $HKY+I+G$    & $ATP6_3, c0775_1, c0775_2, c0775_3$\\ 
    5      & $GTR+I+G$    & $c0036_1, c0036_2, c0036_3$\\ 
    6      & $TrNef+G$    & $COI_2$ \\
    7      & $TrNef+I$    & $COI_3, H3a_2, H3a_3$\\ 
    8      & $HKY$        & $H3a_1, LSU$\\ 
    9      & $TrN+G$      & ITS \\       
  \end{tabular}
\end{table}



\end{document}


Notes about the biology of the species
- EP: discharge Cuv tubules \cite{Bakus1974}, ``relatively very active''
\cite{Bakus1974}, 
- Egypt: tables appear earlier than buttons in juveniles, look for pl.8 fig4-5,
and pl.9 2 in Savigny \cite{Mortensen1902}10.1111/j.1096-3642.1926.tb00326.x

READ \cite{Cutress1996}, seems like there is a lot of information about
ontogenetic changes in the shape of the ossicles.
``Buttons in OM dermis of the 195 mm specimen from Ceylon, however, are
distinctly smaller.'' 


